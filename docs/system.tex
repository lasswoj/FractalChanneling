\part{Fractal channeling - System elements}
\section{Green flows}
\subsection{CRUD -> User can perform crude operations}
\subsubsection{Read configuration}\label{GF:config}
By the authorisation method mentioned in \ref{AUTH:POC} user logs in \\
tickets - NONE
\subsubsection{Log in}\label{GF:LOGIN}
By the authorisation method mentioned in \nameref{AUTH:POC} user logs in \\
tickets - NONE
\subsubsection{Add ticket (optional)}\label{GF:ADD}
By UI mentioned in \nameref{UI:POC} user can add ticket based on template detailed in \nameref{MODEL:POC} 
to a currently selected node. Added tickets go through server mentioned in \nameref{SRV:POC} and land in database mentioned in \nameref{DB:POC}\\
tickets - NONE
\subsubsection{Move to node (optional)}
By UI mentioned in \nameref{UI:POC} user can navigate within datamodel consisting structures defined in \nameref{MODEL:POC}\\
tickets - NONE
\subsubsection{List}
By UI mentioned in \nameref{UI:POC} user can list tickets from current node. This list is cashed so no need for server calls.\\
tickets - NONE
\subsubsection{Put}
By UI mentioned in \nameref{UI:POC} user can update ticket based on template detailed in \nameref{MODEL:POC} 
to replace a part of selected node. Updated tickets go through server mentioned in \nameref{SRV:POC} and land in database mentioned in \nameref{DB:POC}\\
tickets - NONE
\subsubsection{Read}
By UI mentioned in \nameref{UI:POC} user can read tickets. It is cashed operation so no need for server call.\\
tickets - NONE
\subsubsection{Remove}
By UI mentioned in \nameref{UI:POC} user can remove tickets. Removed tickets go through server mentioned in \nameref{SRV:POC} and land in database mentioned in \ref{DB:POC} in a form of a visibility hiding flag. After x time they are replaced for good.\\
tickets - NONE



\newpage
\section{POC phase}
\subsection{User interfaces}\label{UI:POC}
CLI
\subsubsection{Why?}
\begin{enumerate}
    \item Does the job done
    \item Is fast to prototype
    \item Is easy to refactor
\end{enumerate}
\subsubsection{Problems?}
\begin{enumerate}
    \item It's bit unhandy - but for early stage - suffice
\end{enumerate}

\subsection{User authorisation}\label{AUTH:POC}
Local gitignored uname/passwd Toml file.
\subsubsection{Why?}
\begin{enumerate}
    \item Does the job done
    \item Is fast to prototype
    \item Is easy to refactor
    \item For config is just great
\end{enumerate}
\subsubsection{Problems?}
\begin{enumerate}
    \item None - config is just great
\end{enumerate}
\subsection{Database}\label{DB:POC}
Gitignored Json synced
\subsubsection{Why?}
\begin{enumerate}
    \item Does the job done
    \item Is fast to prototype
    \item Is easy to refactor
    \item Makes it easy to extend onto another "databases"
\end{enumerate}
\subsubsection{Problems?}
\begin{enumerate}
    \item To sync it we have to use (for example Dropbox - thanks Poszu) - collisions might occur, but will probably be rare.
\end{enumerate}

\subsection{Data models}\label{MODEL:POC}
Kanban ticket
\subsubsection{Why?}
\begin{enumerate}
    \item Does the job done
    \item Is rather basic
    \item Once this is done it will make the rest work easier
\end{enumerate}
\subsubsection{Problems?}
Nope - there is no problem with it
\subsubsection{Ticket structure}
\begin{itemize}
  \item Primary key       [int]
  \item unique name       [String]
  \item description       [String]
  \item story points      [int]
  \item parent ticket     [OPTIONAL(DROPDOWN(int))]
  \item child ticket      [OPTIONAL(DROPDOWN(int))]
  \item reporter          [DROPDOWN(String)]
  \item asignee           [DROPDOWN(String)]
  \item affected\_modules [list]
  \item relevant\_changes [list]
  \item continued\_in     [optional(int)]
\end{itemize}

\subsection{Server}\label{SRV:POC}
Local Machine
\subsubsection{Why?}
\begin{enumerate}
    \item Does the job done
    \item Is fast to prototype
    \item Is easy to refactor
    \item No need for CI/CD
    \item ...Does the job done!!!
\end{enumerate}
\subsubsection{Problems?}
Well it's gonna work simply as a console application, nothing shiny about it - we code and debug and as we debug we us the app. Stupid simple. 
\newpage

%\setcounter{section}{0}
