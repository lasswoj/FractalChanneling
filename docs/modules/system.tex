
\section{Green flows}
\subsection{User can perform CRUD operations}\label{GF:CRUD}
\subsubsection{Log in}\label{GF:CRUD:LOGIN}
By the \nameref{AUTH:POC} user logs in \\
tickets - \nameref{TICKETS:GF:LOGIN}
\subsubsection{Add}\label{GF:CRUD:ADD}
By \nameref{UI:POC} user can add ticket based on \nameref{MODEL:POC} 
to a currently selected node. Added tickets go through \nameref{SRV:POC} and land in \nameref{DB:POC}\\
tickets - \nameref{TICKETS:GF:ADD}
\subsubsection{Read}\label{GF:CRUD:READ}
By \nameref{UI:POC} user can read tickets. It is cashed operation so no need for server call.\\
tickets - \nameref{TICKETS:GF:READ}
\subsubsection{Update}\label{GF:CRUD:UP}
By \nameref{UI:POC} user can update ticket based on \nameref{MODEL:POC} 
to a currently selected node. Updated tickets go through \nameref{SRV:POC} and replace existing ones in \nameref{DB:POC}\\
tickets - \nameref{TICKETS:GF:UP}
\subsubsection{DELETE}\label{GF:CRUD:DEL}
By \nameref{UI:POC} user can remove tickets. Removed tickets go through server mentioned 
in \nameref{SRV:POC} and land in database mentioned in \ref{DB:POC} in a form of a visibility 
hiding flag. After x time they are replaced for good.\\
tickets - \nameref{TICKETS:GF:DEL}
\subsubsection{Move}\label{GF:CRUD:MOV}
By \nameref{UI:POC} user can navigate through nodes within \nameref{MODEL:POC}\\
tickets - \nameref{TICKETS:GF:MOV}
\subsubsection{List}\label{GF:CRUD:LIST}
By \nameref{UI:POC} user can list tickets from current node. This list is cashed so no 
need for server calls.\\
tickets - \nameref{TICKETS:GF:LIST}

\newpage
\section{POC phase}
\subsection{User interfaces}\label{UI:POC}
CLI
\subsubsection{Why?}
\begin{enumerate}
    \item Does the job done
    \item Is fast to prototype
    \item Is easy to refactor
\end{enumerate}
\subsubsection{Problems?}
\begin{enumerate}
    \item It's bit unhandy - but for early stage - suffice
\end{enumerate}

\subsection{User authorisation}\label{AUTH:POC}
Local gitignored uname/passwd Toml file.
\subsubsection{Why?}
\begin{enumerate}
    \item Does the job done
    \item Is fast to prototype
    \item Is easy to refactor
    \item easy to change it later
\end{enumerate}
\subsection{User config}\label{CONFIG:POC}
Local config Toml file.
\subsubsection{Why?}
\begin{enumerate}
    \item Does the job done
    \item Is fast to prototype
    \item Is easy to refactor
    \item For config is just great
\end{enumerate}
\begin{enumerate}
    \item None - config is just great
\end{enumerate}
\subsection{Database}\label{DB:POC}
Gitignored Json synced by dropbox with symbolic link in ./resources/db1.json
\subsubsection{Why?}
\begin{enumerate}
    \item Does the job done
    \item Is fast to prototype
    \item Is easy to refactor
    \item Makes it easy to extend onto another "databases"
\end{enumerate}
\subsubsection{Problems?}
\begin{enumerate}
    \item To sync it we have to use (for example Dropbox - thanks Poszu) - collisions might occur, but will probably be rare.
\end{enumerate}

\subsection{Data models}\label{MODEL:POC}
Kanban ticket
\subsubsection{Why?}
\begin{enumerate}
    \item Does the job done
    \item Is rather basic
    \item Once this is done it will make the rest work easier
\end{enumerate}
\subsubsection{Problems?}
Nope - there is no problem with it
\subsubsection{Ticket structure}
\begin{itemize}
  \item primary\_key        [int]
  \item unique\_name        [String]
  \item description         [String]
  \item story\_points       [int]
  \item parent\_ticket      [OPTIONAL(int)]
  \item child\_ticket       [OPTIONAL(int)]
  \item reporter            [String]
  \item asignee             [String]
  \item affected\_modules   [list(String)]
  \item affected\_steps     [list(String)]
  \item relevant\_changes   [list(String)]
  \item continued\_in       [OPTIONAL(int)]
\end{itemize}

\subsection{Server}\label{SRV:POC}
Local Machine
\subsubsection{Why?}
\begin{enumerate}
    \item Does the job done
    \item Is fast to prototype
    \item Is easy to refactor
    \item No need for CI/CD
    \item ...Does the job done!!!
\end{enumerate}
\subsubsection{Problems?}
Well it's gonna work simply as a console application, nothing shiny about it - we code and debug and as we debug we us the app. Stupid simple. 
\newpage

%\setcounter{section}{0}

\section{Tickets listings}
\labelText{\href{run:../resources/db1.json}{GF::CRUD::\nameref*{GF:CRUD:LOGIN}}}{TICKETS:GF:LOGIN}\\
\labelText{\href{run:../resources/db1.json}{GF::CRUD::\nameref*{GF:CRUD:ADD}}}{TICKETS:GF:ADD}\\
\labelText{\href{run:../resources/db1.json}{GF::CRUD::\nameref*{GF:CRUD:READ}}}{TICKETS:GF:READ}\\
\labelText{\href{run:../resources/db1.json}{GF::CRUD::\nameref*{GF:CRUD:UP}}}{TICKETS:GF:UP}\\
\labelText{\href{run:../resources/db1.json}{GF::CRUD::\nameref*{GF:CRUD:DEL}}}{TICKETS:GF:DEL}\\
\labelText{\href{run:../resources/db1.json}{GF::CRUD::\nameref*{GF:CRUD:MOV}}}{TICKETS:GF:MOV}\\
\labelText{\href{run:../resources/db1.json}{GF::CRUD::\nameref*{GF:CRUD:LIST}}}{TICKETS:GF:LIST}\\