
\section{Green flows}
\subsection{CRUD operations}\label{GF:CRUD}
User can perform typical crud and allso move and list operations.
\subsubsection{Log in}\label{GF:CRUD:LOGIN}
By the \nameref{AUTH:POC} user logs in \\
tickets - \nameref{TICKETS:GF:LOGIN}
\subsubsection{Add}\label{GF:CRUD:ADD}
By \nameref{UI:POC} user can add ticket based on \nameref{MODEL:POC} 
to a currently selected node. Added tickets go through \nameref{SRV:POC} and land in \nameref{DB:POC}\\
tickets - \nameref{TICKETS:GF:ADD}
\subsubsection{Read}\label{GF:CRUD:READ}
By \nameref{UI:POC} user can read tickets. It is cashed operation so no need for server call.\\
tickets - \nameref{TICKETS:GF:READ}
\subsubsection{Update}\label{GF:CRUD:UP}
By \nameref{UI:POC} user can update ticket based on \nameref{MODEL:POC} 
to a currently selected node. Updated tickets go through \nameref{SRV:POC} and replace existing ones in \nameref{DB:POC}\\
tickets - \nameref{TICKETS:GF:UP}
\subsubsection{DELETE}\label{GF:CRUD:DEL}
By \nameref{UI:POC} user can remove tickets. Removed tickets go through server mentioned 
in \nameref{SRV:POC} and land in database mentioned in \ref{DB:POC} in a form of a visibility 
hiding flag. After x time they are replaced for good.\\
tickets - \nameref{TICKETS:GF:DEL}
\subsubsection{Move}\label{GF:CRUD:MOV}
By \nameref{UI:POC} user can navigate through nodes within \nameref{MODEL:POC}\\
tickets - \nameref{TICKETS:GF:MOV}
\subsubsection{List}\label{GF:CRUD:LIST}
By \nameref{UI:POC} user can list tickets from current node. This list is cashed so no 
need for server calls.\\
tickets - \nameref{TICKETS:GF:LIST}

\newpage
\section{System description}
\subsection{User interfaces}\label{UI:POC}

\subsubsection{CLI}
discord discussions:\\
\href{https://discord.com/channels/892473074434310144/896321419498098689}{initial discussion}\\
Most important features
\begin{enumerate}
    \item allows to interact with system using instructions from command line
    \item enables \nameref{GF:CRUD}
\end{enumerate}

\subsection{User authorisation and config}\label{AUTH:POC}\label{CONFIG:POC}
discord discussions:
\begin{itemize}
    \item \href{https://discord.com/channels/892473074434310144/896357478600671272}{initial discussion- authorization}
    \item \href{https://discord.com/channels/892473074434310144/896360600542789642}{initial discussion- config}
\end{itemize}
Username and password is stroed in apopreate form in local gitignored configuration Toml file. 
Program reads this file in the beginnig of the project. 
\subsection{Database}\label{DB:POC}
\subsubsection{local Json}
Gitignored Json synced by dropbox with symbolic link in ./resources/db1.json.\\
how to setup(\href{https://discord.com/channels/892473074434310144/895768675146154004/895772874613346354}{discord discussion}):
\begin{enumerate}
    \item Send mail to Wojtek (i will share folder "sync")
    \item Get dropbox
    \item Get dropbox local folder
    \item Check if you have the shared folder "synced" with db.json in it
    \item In "resources" project folder create symbolic link (shortcut) with name "db1.json" pointing to it.
    \item Drink
\end{enumerate}

\href{https://discord.com/channels/892473074434310144/892481179624824843}{initial discussion}

\subsection{Data models}\label{MODEL:POC}
\subsubsection{Kanban ticket}
Ticket structure
\begin{itemize}
  \item primary\_key        [int]
  \item unique\_name        [String]
  \item description         [String]
  \item story\_points       [int]
  \item parent\_ticket      [OPTIONAL(int)]
  \item child\_tickets      [OPTIONAL(list(int))]
  \item reporter            [String]
  \item asignee             [String]
  \item affected\_modules   [list(String)]
  \item affected\_steps     [list(String)]
  \item relevant\_changes   [list(String)]
  \item continued\_in       [OPTIONAL(int)]
  \item status              [String]
\end{itemize}
\href{https://discord.com/channels/892473074434310144/892483718495731753}{initial discussion}

\subsection{Server}\label{SRV:POC}
\subsubsection{Local Machine}
It is a separated program running on localhost. \\
\href{https://discord.com/channels/892473074434310144/896382199736172564}{initial discussion}\\
Most important features
\begin{enumerate}
    \item It can read "srvconfig.toml".
    \item on any request regardless of input it returns content of servconfig.toml
\end{enumerate}


%\setcounter{section}{0}

\section{Tickets listings}
\labelText{\href{run:../resources/db1.json}{GF::CRUD::\nameref*{GF:CRUD:LOGIN}}}{TICKETS:GF:LOGIN}\\
\labelText{\href{run:../resources/db1.json}{GF::CRUD::\nameref*{GF:CRUD:ADD}}}{TICKETS:GF:ADD}\\
\labelText{\href{run:../resources/db1.json}{GF::CRUD::\nameref*{GF:CRUD:READ}}}{TICKETS:GF:READ}\\
\labelText{\href{run:../resources/db1.json}{GF::CRUD::\nameref*{GF:CRUD:UP}}}{TICKETS:GF:UP}\\
\labelText{\href{run:../resources/db1.json}{GF::CRUD::\nameref*{GF:CRUD:DEL}}}{TICKETS:GF:DEL}\\
\labelText{\href{run:../resources/db1.json}{GF::CRUD::\nameref*{GF:CRUD:MOV}}}{TICKETS:GF:MOV}\\
\labelText{\href{run:../resources/db1.json}{GF::CRUD::\nameref*{GF:CRUD:LIST}}}{TICKETS:GF:LIST}\\